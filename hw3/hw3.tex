% Homework 3 - CS386
% Russell Miller Winter 2011

\documentclass{article}
\usepackage{anysize}
\usepackage{wasysym}

\marginsize{2cm}{2cm}{2cm}{2cm}

\title{CS386 Homework 3}
\author{Russell Miller}
\date{\today}

% Relational Algebra
% SELECT [columns] (aka PROJECT): $\pi$ (π)
% FROM [rows] (aka SELECTION): $\sigma$ (σ)
% RENAME: $\rho$ (ρ)
% JOIN: \Bowtie (⨝) (requires \usepackage{wasysym})
%    note: R \Bowtie_{c} S = \sigma_{c} (R \times S)

\begin{document}

\maketitle

\section*{Part 1}
\textbf{Turn in the SQL query, the first five rows of the query answer, and the number of rows
in your query answer.}
\begin{enumerate}

\item
\textbf{Using the spy database, write an SQL query to find the names of teams who have failed
exactly two missions.\\}
SELECT team.name\\
FROM mission,team\\
WHERE mission.team\_id=team.team\_id\\
AND mission.mission\_status='failed'\\
GROUP BY team.name\\
HAVING COUNT(*) = 2;\\
\begin{tabular}{l}
 name      \\
\hline
 Roadkill\\
 Vikings\\
 Blaster\\
 Failsafe\\
 ShowBiz\\
\end{tabular}
\\13 rows.\\

\item
\textbf{Using the library database, list the authors that have published the largest number
of books.\\}
SELECT author.firstname,author.middlename,author.lastname\\
FROM author,relauth\\
WHERE author.authorid=relauth.authorid\\
GROUP BY author.firstname,author.middlename,author.lastname\\
HAVING COUNT(*) = (SELECT max(COUNT)\\
\hspace*{4 cm}      FROM (SELECT COUNT(*)\\
\hspace*{6 cm}            FROM author,relauth,book\_description\\
\hspace*{6 cm}            WHERE author.authorid=relauth.authorid\\
\hspace*{6 cm}            AND relauth.bookdescid=book\_description.bookdescid\\
\hspace*{6 cm}            GROUP BY author.authorid) AS totals);\\
\begin{tabular}{l|l|l}
 firstname & middlename &  lastname  \\
\hline
 DEMETRIOS & G          & LAINIOTIS\\
 KRITHI    &            & RAMAMRITHA\\
 D.        &            & BAKER\\
 THOMAS    & C          & MCINTIRE\\
 JOHN      &            & MCCONNELL\\
\end{tabular}
\\5 rows.\\

\pagebreak

\item
\textbf{Using the spy databse, are the following two SQL qeueries equivalent? Describe the
two queries in English and explain how they are different.\\}
No. They are different queries. (a) uses a natural join on the Team table and the Mission
table. This means equality is specified on all fields with the same name. Since Team and
Mission both have a 'name' column, the two would have to be equal in order for the row to
show up in the query answer. (a) lists the successful missions where the team name is the
same as the mission name. (b) lists the successful missions for each team, along with the
team name.\\

\item
\textbf{List the agents who were assigned to at least one 'failed' mission. Write this
query in two versions, one using EXISTS and another using IN.\\}

SELECT a.first,a.last\\
FROM agent a\\
WHERE EXISTS (SELECT DISTINCT agent.agent\_id\\
\hspace*{3 cm}FROM agent,teamrel,team,mission\\
\hspace*{3 cm}WHERE agent.agent\_id=teamrel.agent\_id\\
\hspace*{3 cm}AND teamrel.team\_id=team.team\_id\\
\hspace*{3 cm}AND team.team\_id=mission.team\_id\\
\hspace*{3 cm}AND mission\_status='failed'\\
\hspace*{3 cm}AND agent.agent\_id=a.agent\_id\\
\hspace*{3 cm}GROUP BY agent.agent\_id\\
\hspace*{3 cm}HAVING COUNT(*) > 0);\\
\begin{tabular}{l|l}
first   &      last       \\
\hline
 Nick      & Black\\
 Mathew    & Cohen\\
 Jim       & Cowan\\
 George    & Fairley\\
 George    & Jones\\
\end{tabular}
\\221 rows.\\

SELECT a.first,a.last\\
FROM agent a\\
WHERE a.agent\_id IN (SELECT DISTINCT agent.agent\_id\\
\hspace*{4 cm}        FROM agent,teamrel,team,mission\\
\hspace*{4 cm}        WHERE agent.agent\_id=teamrel.agent\_id\\
\hspace*{4 cm}        AND teamrel.team\_id=team.team\_id\\
\hspace*{4 cm}        AND team.team\_id=mission.team\_id\\
\hspace*{4 cm}        AND mission\_status='failed'\\
\hspace*{4 cm}        GROUP BY agent.agent\_id\\
\hspace*{4 cm}        HAVING COUNT(*) $>$ 0);\\
\begin{tabular}{l|l}
   first   &      last\\       
\hline
 Nick      & Black\\
 Mathew    & Cohen\\
 Jim       & Cowan\\
 George    & Fairley\\
 George    & Jones\\
\end{tabular}
\\221 rows.\\

\pagebreak

\item
\textbf{List the agents that have skills of 'Firearms' or 'Biologist' using the IN clause.\\}
SELECT a.first,a.last\\
FROM agent a\\
WHERE a.agent\_id IN (SELECT agent.agent\_id\\
\hspace*{4 cm}       FROM agent,skillrel,skill\\
\hspace*{4 cm}       WHERE agent.agent\_id=skillrel.agent\_id\\
\hspace*{4 cm}       AND skillrel.skill\_id=skill.skill\_id\\
\hspace*{4 cm}       AND (skill.skill='Firearms' OR skill.skill='Biologist'));\\
\begin{tabular}{l|l}
first     &    last     \\
\hline
 Nick         & Black\\
 Chris        & Leen\\
 Kristin      & Moody\\
 Nick         & Steere\\
 Pete         & Consel\\
\end{tabular}
\\84 rows.\\

\item
\textbf{Rewrite the query described in question 5 without using IN.\\}
SELECT agent.first,agent.last\\
FROM agent,skillrel,skill\\
WHERE agent.agent\_id=skillrel.agent\_id\\
AND skillrel.skill\_id=skill.skill\_id\\
AND (skill.skill='Firearms' or skill.skill='Biologist');\\
\begin{tabular}{l|l}
first     &    last     \\
\hline
 Nick         & Steere\\
 Helen        & Hermansky\\
 Pete         & Pickering\\
 Robert       & Smith\\
 George       & Berkling\\
\end{tabular}
\\84 rows.\\

\item
\textbf{Write a relational algebra query that is equivalent to the query described in question 5.\\}
$\pi_{agent.first,agent.last}($\\
$agent \Bowtie_{agent.agent\_id=skillrel.agent\_id} ($\\
$skillrel \Bowtie_{skillrel.skill\_id=skill.skill\_id} ($\\
$\sigma_{skill.skill='Firearms'\; \vee\; skill.skill='Biologist'}skill)))$\\

\end{enumerate}

\end{document}
