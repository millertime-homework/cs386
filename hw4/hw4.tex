% Homework 4 - CS386(late submission)
% Russell Miller Winter 2011

\documentclass{article}
\usepackage{anysize}
\usepackage{wasysym}

\marginsize{2cm}{2cm}{2cm}{2cm}

\title{CS386 Homework 4\\
(late submission)}
\author{Russell Miller}
\date{\today}

% Relational Algebra
% SELECT [columns] (aka PROJECT): $\pi$ (π)
% FROM [rows] (aka SELECTION): $\sigma$ (σ)
% RENAME: $\rho$ (ρ)
% JOIN: \Bowtie (⨝) (requires \usepackage{wasysym})
%    note: R \Bowtie_{c} S = \sigma_{c} (R \times S)

\begin{document}

\maketitle

\begin{enumerate}
\item
\textbf{Prove these two expressions are equivalent:\\}
$\sigma_{sid > 0}(\sigma_{sid < 207}(\pi_{sid}(\sigma_{color=blue}(\pi_{sid,color}(
 reserves \Bowtie boats)))))$\\
$\sigma_{sid > 0 \wedge sid < 207}(\pi_{sid}(\sigma_{color=blue}(
 boats) \Bowtie reserves))$\\

\begin{quote}
Because the selection $\sigma_{color=blue}$ only involves an attribute retained by the
projection $\pi_{sid,color}$, the two can be commuted.\\
$\sigma_{sid > 0}(\sigma_{sid < 207}(\pi_{sid}(\pi_{sid,color}(\sigma_{color=blue}(
 reserves \Bowtie boats)))))$\\
\\
The join $reserves \Bowtie boats$ can be commuted.\\
$\sigma_{sid > 0}(\sigma_{sid < 207}(\pi_{sid}(\pi_{sid,color}(\sigma_{color=blue}(
 boats \Bowtie reserves)))))$\\
\\
Because the selection $\sigma_{color=blue}$ involves only color, which is not an attribue
of reserves, it can be commuted with the join $boats \Bowtie reserves$.\\
$\sigma_{sid > 0}(\sigma_{sid < 207}(\pi_{sid}(\pi_{sid,color}(\sigma_{color=blue}(
 boats) \Bowtie reserves))))$\\
\\
The cascading projections $\pi_{sid}(\pi_{sid,color}(R))$ are equivalent to the final
projection.\\
$\sigma_{sid > 0}(\sigma_{sid < 207}(\pi_{sid}(\sigma_{color=blue}(
 boats) \Bowtie reserves)))$\\
\\
The cascading selections $\sigma_{sid > 0}(\sigma_{sid < 207}(R))$ are equivalent to
a single conjunction selection.\\
$\sigma_{sid > 0 \wedge sid < 207}(\pi_{sid}(\sigma_{color=blue}(
 boats) \Bowtie reserves))$\\
\end{quote}

\item
\textbf{Rewrite this query with SUB-SELECT instead of WHERE.\\}
SELECT agent\_id\\
FROM skillrel\\
GROUP BY agent\_id HAVING count(skill\_id) $>$ 4;\\
\begin{quote}
SELECT DISTINCT agent\_id\\
FROM skillrel sr1\\
WHERE 4 $<$ (SELECT count(skill\_id)\\
\hspace*{2cm}FROM skillrel sr2\\
\hspace*{2cm}WHERE sr1.agent\_id=sr2.agent\_id);\\
\\
The query response was 115 rows.
\end{quote}

\end{enumerate}
\end{document}
