% Homework 7 - CS386
% Russell Miller Winter 2011

\documentclass{article}
\usepackage{anysize}
\usepackage{wasysym}
\usepackage{graphicx}

\marginsize{2cm}{2cm}{2cm}{2cm}

\title{CS386 Homework 7}
\author{Russell Miller}
\date{\today}

\begin{document}

\maketitle

\section*{Part 1}
\begin{enumerate}
\item
\begin{description}
\item[a.]
This would be closer to the case of (M+N) because if there is
a match between Patient.SSN and Provider.SSN there can only be
one row returned, since it is a unique value for each table.\\

\item[b.]
This would be closer to the case of (M*N) because it is possible
for every Student to be male, and every Staff to be male, in which
case the number of rows would be the number of Staff times the number
of Students.\\

\end{description}

\item
\begin{description}
\item[a.]
Using an external sort the time would be 2*M, or just 2.

\item[b.]
The page-oriented nested loop join requires that each table be scanned
M times, and the total cost is M+M*N -- in this case 2 + 2N.

\end{description}

\item
\begin{description}
\item[a.]
No.

\item[b.]
Clustering indexes improves sort time, but it cannot improve join time.
It is still a sequential read for each table, and because of the indexes
actually takes longer.

\end{description}
\end{enumerate}

\section*{Part 2A}
\begin{enumerate}
\item The result of the query would be sorted by agent\_id.
\item This would be sorted by salary.
\item Because those are the clustered indexes.
\end{enumerate}

\section*{Part 2B}
\begin{enumerate}
\item
\begin{description}
\item[a.]
For a) the expected cost is 54.60..54.61.
For b) the expected cost is 54.60..54.61.
So they have the same estimated cost.

\item[b.]
Query a) was faster clocking at 5.242 ms. Query b) took 5.367 ms.

\item[c.]
Postgres chose to use Hash join for both queries.

\item[d.]
It doesn't look like it's using clustered indexes.

\end{description}

\item
\begin{description}
\item[a.]
For a) the expected cost is 48.24..48.25.
For b) the expected cost is 48.24..48.25.

\item[b.]
Query b) was faster, clocking 3.822 ms. Query a) timed 3.854 ms.

\item[c.]
They were both Hash joins.

\item[d.]
There are still no clustered indexes.

\end{description}

\item
\begin{description}
\item[a.]
For a) the expected cost is 48.24..48.25.
For b) the expected cost is 29.12..29.13.

\item[b.]
Query b) was faster, clocking 2.141 ms. Query a) timed 3.903 ms.

\item[c.]
They were both Hash joins.

\item[d.]
Query b) was clustered on agent\_salary, but Query a) was not.

\end{description}

\item
For many of those queries, none of the rows got filtered out.

\end{enumerate}

\section*{Part 3}
\begin{enumerate}
\item
\begin{description}
\item[a.]
Yes, the constraint $salary > 0$ is pushed into the inner query.

\item[b.]
It seems like it would be more efficient because less rows would be
returned to the outer query.

\item[c.]
No.

\end{description}

\item
\begin{description}
\item[a.]
Yes, the constraint $salary < 56000$ is pushed into the inner query.

\item[b.]
This time it's definitely more efficient.

\item[c.]
Yes, it is clustered on salary.

\end{description}

\item
\begin{description}
\item[a.]
Yes, the contraint $salary < 56000$ is pushed into the first inner query
alongside $salary > 55000$.

\item[b.]
I would bet that this is more efficient because of the clustered index.

\item[c.]
Yes.

\end{description}

\end{enumerate}

\end{document}
